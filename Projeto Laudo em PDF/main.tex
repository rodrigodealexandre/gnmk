%START package dependency
\documentclass[12pt]{article}
\usepackage{pst-all,amssymb,graphicx,marvosym,array,fancyhdr,amsmath,multirow,pdflscape,rotating}
\usepackage{color}
\usepackage[utf8]{inputenc}
\usepackage[default]{lato}
\usepackage{cmbright}
\usepackage[T1]{fontenc}
\usepackage{graphicx}
\usepackage[most]{tcolorbox}
\usepackage{lastpage}
\usepackage[yyyymmdd,hhmmss]{datetime}
%\usepackage{eforms}
%END package dependency

%START page properties
\def\dfrac#1#2{\displaystyle{\frac{#1}{#2}}}
\paperheight=297mm
\voffset=-1in
\topmargin=0.7cm
\headheight=3.5cm
\headsep=0.5cm
\textheight=19.7cm
\paperwidth=210mm
\hoffset=-1in
\oddsidemargin=1cm
\evensidemargin=2.5cm
\textwidth=19cm
\marginparsep=0cm
\marginparwidth=0cm
\setlength{\parindent}{0pt}
%END page properties

%START color types
\definecolor{airforceblue}{rgb}{0.36, 0.54, 0.66}
\definecolor{coolblack}{rgb}{0.0, 0.18, 0.39}
%END color types

% START Table creation code
\newtcolorbox{tableheaderbox}[1][]{colback={airforceblue!20!white}, colframe={airforceblue!20!white}, coltext={coolblack}, sharp corners, arc=2mm, rounded corners=north, height=15mm, width=30mm, valign=center, left=1mm, top=1mm, bottom=1mm, right=1mm, boxsep=0mm, nobeforeafter, halign=center, box align=center, equal height group=tableheader, fontupper=\footnotesize\bfseries} 
\newtcolorbox{contentbox}[1][]{
 enhanced jigsaw, boxrule=0pt, colback=white, height=10mm, width=30mm, valign=center, left=1mm, top=1mm,
bottom=1mm, right=1mm, boxsep=0mm, nobeforeafter, halign=center, box align=center, fontupper=\footnotesize, #1} % use "borderline vertical={1pt}{0pt}{airforceblue!20!white}" for vertical lines
\newtcolorbox{rulerbox}[1][]{width=30mm, size=minimal, height=0pt, enhanced jigsaw, 
  sharp corners, borderline horizontal={0.5pt}{0pt}{airforceblue!20!white}, nobeforeafter, #1}
\newcommand{\fakehline}[1][]{\begin{rulerbox}[#1] \end{rulerbox}}
% END Table creation code

\begin{document}

%START Header and Footer
\fancyhf{}
\pagestyle{fancy}
\lhead{\footnotesize \sc \begin{tabular}{m{12cm} m{6cm}} \includegraphics[scale=0.32]{genomika2.png} \\
{\fontseries{eb}\selectfont \textbf{Cliente:}} *NAME* & {\fontseries{eb}\selectfont \textbf{Sexo:}} *SEX* \\
{\fontseries{eb}\selectfont \textbf{Unidade:}} *UNITY* & {\fontseries{eb}\selectfont \textbf{Nascimento:}} *BIRTH* \\
{\fontseries{eb}\selectfont \textbf{Atendimento:}} *ATTENDANCE* & {\fontseries{eb}\selectfont \textbf{Data Aten.:}} *ATTEN.DATE* \\
{\fontseries{eb}\selectfont \textbf{Solicitante:}} *REQUESTER* & {\fontseries{eb}\selectfont \textbf{Local de Coleta:}} *LOCAL* \\
{\fontseries{eb}\selectfont \textbf{Material:}} *MATERIAL* & {\fontseries{eb}\selectfont \textbf{Data de Coleta:}} *COLL.DATE* \\
\end{tabular}}
\renewcommand{\headrulewidth}{0pt}
%END Header

%START Footer
\renewcommand{\footrulewidth}{0.4pt}
\lfoot{\scriptsize \begin{tabular}{m{11cm} m{7.5cm}}
{\fontseries{eb}\selectfont \textbf{Responsável Técnico: Dr. João Bosco - CRM 12563 PE}} & {\fontseries{eb}\selectfont \textbf{Genomika Diagnósticos S/A - CRM 2374/PE}} \\
R. Senador José Henrique, 224, Salas 1301/1303, Emp. Alfred Nobel, Ilha do Leite & www.genomika.com.br \\
Recife - PE CEP:50070 460, Telefone: +55 81 3125-0505 
\end{tabular}
\begin{tabular}{m{18cm}}
{A interpretação de qualquer teste laboratorial com finalidade diagnóstica ou prognóstica compete exclusivamente ao seu médico e sua análise depende da avaliação conjunta de sua história médica, dados clínicos e resultados de outros exames.}
\end{tabular}}
\rfoot{\\[0.15cm] \scriptsize{\today \ \currenttime \ Pág.:\thepage \ de \pageref{LastPage}}}
%END Footer

%START Title content
\begin{center}
{\selectfont\fontseries{eb}\color{coolblack}\LARGE\bfseries *TITLE*}
\end{center}
%END Title content

%START Clinical Summary content
{\selectfont\fontseries{eb}\color{coolblack}\normalsize\bfseries Resumo Clínico}\hfill\break
*CLIN.SUMMURY*
\bigskip
%END Clinical Summary content

%START Summary content
\tcbset{colback=airforceblue!20!white,colframe=airforceblue!20!white}
\begin{tcolorbox}
\textbf{Resumo dos Resultados}\hfill\break
*SUMMARY*
\end{tcolorbox}
\bigskip
%START Summary content


%START Gene Summary content
{\selectfont\fontseries{eb}\color{coolblack}\normalsize\bfseries Genes Analisados}\hfill\break
*GENES*
\bigskip
%END Gene Summary content


%\tcbset{myone/.style={colframe=airforceblue!20!white,halign=center,valign=bottom,center title, equal height group=nobefaf,width=\linewidth/6,nobeforeafter}}

%\begin{tcolorbox}[myone,title={\color{coolblack}\normalsize\bfseries Gene/ Transcrito},before app={\selectfont\fontseries{eb} \color{coolblack}\normalsize \bfseries Variantes Clinicamente Associadas e %Achados Adicionais\\[4pt]}]Box 1\end{tcolorbox}%
%\begin{tcolorbox}[myone,title={\color{coolblack}\normalsize\bfseries Variante}]Box 2\end{tcolorbox}%
%\begin{tcolorbox}[myone,title={\color{coolblack}\normalsize\bfseries Exon}]Box 3\end{tcolorbox}%
%\begin{tcolorbox}[myone,title={\color{coolblack}\normalsize\bfseries Frequência na População}]Box 4\end{tcolorbox}%
%\begin{tcolorbox}[myone,title={\color{coolblack}\normalsize\bfseries Zigosidade}]Box 5\end{tcolorbox}%
%\begin{tcolorbox}[myone,title={\color{coolblack}\normalsize\bfseries Classificação}]Box 6\end{tcolorbox}
%\bigskip\bigskip



% START Variant table
{\selectfont\fontseries{eb} \color{coolblack}\normalsize \bfseries Variantes Clinicamente Associadas e Achados Adicionais}
\begin{center}
% START Headerboxes
\begin{tcbraster}[raster columns=6,raster column skip = -0.5pt, raster row skip = 0pt]
\begin{tableheaderbox} Gene/Transcrito \end{tableheaderbox} \begin{tableheaderbox} Variante \end{tableheaderbox} 
\begin{tableheaderbox} Exon \end{tableheaderbox} \begin{tableheaderbox} Frequência na População \end{tableheaderbox} \begin{tableheaderbox} Zigosidade \end{tableheaderbox} \begin{tableheaderbox} Classificação \end{tableheaderbox} 

% END Headerboxes

% START 1st row
\begin{contentbox} *GENE1* \end{contentbox} \begin{contentbox} *NOTATION1* \end{contentbox}
\begin{contentbox} *EXON1* \end{contentbox} \begin{contentbox} *FREQ1* \end{contentbox} \begin{contentbox} *ZIGO1* \end{contentbox} \begin{contentbox} *CLASS1* \end{contentbox}
\fakehline\fakehline\fakehline\fakehline\fakehline\fakehline
% END 1st row

% START 2nd row
\begin{contentbox} *GENE2* \end{contentbox} \begin{contentbox} *NOTATION2* \end{contentbox}
\begin{contentbox} *EXON2* \end{contentbox} \begin{contentbox} *FREQ2* \end{contentbox} \begin{contentbox} *ZIGO2* \end{contentbox} \begin{contentbox} *CLASS2* \end{contentbox}
\fakehline\fakehline\fakehline\fakehline\fakehline\fakehline
% END 2nd row

% START 3rd row
\begin{contentbox} *GENE3* \end{contentbox} \begin{contentbox} *NOTATION3* \end{contentbox}
\begin{contentbox} *EXON3* \end{contentbox} \begin{contentbox} *FREQ3* \end{contentbox} \begin{contentbox} *ZIGO3* \end{contentbox} \begin{contentbox} *CLASS3* \end{contentbox}
\fakehline\fakehline\fakehline\fakehline\fakehline\fakehline
% END 3rd row


\end{tcbraster}
\end{center}
\bigskip
% END Variant table

%START INTERPRETATION content
{\selectfont\fontseries{eb}\color{coolblack}\normalsize\bfseries Interpretação}\hfill\break
Foi encontrada a variante p.(L747\textunderscore T751del) no exon 19 do gene EGFR,  que resulta numa deleção in-frame de 5 aminoácidos da posição 747 a 751 (LREAT). De acordo com a literatura, as deleções neste exon ocorrem em 48 dos casos de tumores positivos para mutações em EGFR. Esta variante está associada com sensibilidade aumentada aos inibidores de EGFR tirosina-quinase. Não foram encontradas outras alterações significativas em outros exons do mesmo gene. sem mutações ativadoras em NRAS são potencialmente sensíveis à terapia com anticorpos anti-EGFR.
\bigskip
%END INTERPRETATION content

%START RECOMMENDATION content
{\selectfont\fontseries{eb}\color{coolblack}\normalsize\bfseries Recomendações}\hfill\break
*RECOMMENDATION*

\fakehline\fakehline\fakehline\fakehline\fakehline\fakehline
%END RECOMMENDATION content

%START METHODOLOGY content
{\selectfont\fontseries{eb}\color{coolblack}\footnotesize\bfseries Metodologia}\hfill\break
\footnotesize{*METHODOLOGY*}\hfill\break
%END METHODOLOGY content
%START LIMITATIONS content
{\selectfont\fontseries{eb}\color{coolblack}\footnotesize\bfseries Limitações}\hfill\break
\footnotesize{*LIMITATIONS*}\hfill\break
%END LIMITATIONS content
%START COMMENTS content
{\selectfont\fontseries{eb}\color{coolblack}\footnotesize\bfseries Observações}\hfill\break
\footnotesize{*COMMENTS*}\hfill\break
%END COMMENTS content
%START RECTIFICATION content
\footnotesize{*RECTIFICATION*}\hfill\break
%END RECTIFICATION content
%START REFERENCES content
{\selectfont\fontseries{eb}\color{coolblack}\footnotesize\bfseries Referências}\hfill\break
\footnotesize{*REFERENCES*}\hfill\break
\bigskip
%END REFERENCES content

Texto texto texto \\
Texto texto texto \\
Texto texto texto \\
Texto texto texto \\

%START SIGNATURE content
\small{Liberado por *SIGNATUREBY*, em *SIGNATUREDATE*.}\hfill\break
\small{Certificação Digital:}\hfill\break
%\sigField{My signature}{5cm}{3cm}
%END SIGNATURE content

\newpage

Quebra de linha teste

\end{document}